% LaTeX Template for writing mathematics papers.
\documentclass[reqno]{amsart}
\usepackage[all,ps,cmtip]{xy}

% Import the style file
\usepackage{algebraPaper}

% Bibliography
\usepackage[]{biblatex}   % First argument is for the bibliography style.
\bibliography{algebra.bib}

\begin{document}

% Author details.
\author{J. Borger}
\address{Mathematical Sciences Institute, Australian National University, Australia}
\email{james.borger@anu.edu.au}

\author{R. Culling}
\address{Te Kura P\={a}ngarau, Te Whare W\={a}nanga o Waitaha, Aotearoa}
\email{robert.culling@canterbury.ac.nz}

% Paper title
\title{The \'{E}tale Fundamental Group $\bn$}


\begin{abstract}

\end{abstract}

\maketitle

\tableofcontents

\section{Introduction}



\vspace{1em}
{\bf Acknowledgments:}
Most of this work was done during Robert Culling's time as James Borger's PhD student at the Australian National University, from 2015 through 2019. This work is presented in much more detail in Culling's thesis which can be found online \cite{}. Culling would like to acknowledge the support of the Mathematical Sciences Institute at ANU and the support from the Australian People through the Australian Government Research Training Program Stipend and Scholarship. Without this kindness the work would not have been completed.

\newpage
\section{Commutative Algebra of the Natural Numbers}

We present the following section on the commutative algebra of the natural numbers for sake of completeness. We follow the presentation of Bertrand To{\"e}n and Michel Vaqui{\'e} \cite{toenvaquie} and James Borger \cite{jbwittsemipos}. The reader may consult these authors for further details. One may also like to consult Johnathon Golan \cite{golan} for a different perspective on monoids and semirings.

\begin{defn}
A commutative monoid is a triple $(M,+_{M},0_{0})$ which consists of: a set, $M$; a commutative and associative binary operation $+:M\times M \rightarrow M$ called addition, and the distinguished element $0 \in M$ such that $+(0,m) = m$ called the identity. If $(M,+_{M},0_{M})$ and $(N,+_{N},0_{N})$ are commutative monoids, then a homomorphism is a function of sets $\varphi:M \rightarrow N$ which maps $\varphi(0_{M}) = 0_{N}$ and commutes with the binary operations $+_{M}$ and $+_{N}$.
\end{defn}

We will denote a commutative monoid by its set and drop the subscripts from the binary operation and identity. Similarly we will write $a+b$ instead of $+(a,b)$. Throughout this paper commutative monoids will be referred to as $\bn$-modules and the category of $\bn$-modules will be denoted $\nmod$.

The category $\nmod$ has products and coproducts. If $I$ is a set indexing a family of $\bn$-modules $\{M_{i}\}_{I}$, then the set theoretic product $\prod_{I}M_{i}$ with component wise addition and the distinguished element $0 = (...,0,...)$ is the product in $\nmod$. It has the sub-$\bn$-module $\bigoplus_{I}M_{i} \subseteq \prod_{I}M_{i}$ of elements that have only finitely many elements different from 0, this forms the coproduct of the family in $\nmod$.

If $M,N$ are $\bn$-modules, then the set $\homnmod(M,N)$ is an $\bn$-module under point-wise addition of homorphisms. Thus for a fixed $\bn$-module $M$, there is an endofunctor $\homnmod(M, -): \nmod \rightarrow \nmod$ on the category of $\bn$-modules. This functor has a left adjoint, $M\otimes -$, which enduces a symmetric monoidal structure on $\nmod$ with identity $\bn$.

\begin{defn}

A commutative semiring is defined to be a monoid object in the monoidal category $(\nmod,\otimes)$. 

\end{defn}





















\newpage
\section{Arithmetic Geometry of the Natural Numbers}

\newpage
\section{Galois Category of \'{E}tale Covers}

\newpage
\section{Non-Negative Reals and Real Number Fields}

\newpage
\section{Natural Numbers}

\newpage
\section{Non-Trivial Quadratic Finite \'{E}tale Cover}

\newpage
\printbibliography
\vspace{2em}
\end{document}

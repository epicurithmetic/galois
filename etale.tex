% LaTeX Template for writing mathematics papers.
\documentclass[reqno]{amsart}
\usepackage[all,ps,cmtip]{xy}

% Import the style file
\usepackage{algebraPaper}

% Bibliography
\usepackage[]{biblatex}   % First argument is for the bibliography style.
\bibliography{algebra.bib}

\begin{document}

% Author details.
\author{J. Borger}
\address{Mathematical Sciences Institute, Australian National University, Australia}
\email{james.borger@anu.edu.au}

\author{R. Culling}
\address{Te Kura P\={a}ngarau, Te Whare W\={a}nanga o Waitaha, Aotearoa}
\email{robert.culling@canterbury.ac.nz}

% Paper title
\title{The \'{E}tale Fundamental Group of $\bn$-Schemes}


\begin{abstract}

We do some cool stuff. You should read this.

\end{abstract}

\maketitle

\tableofcontents

\section{Introduction}


\newpage
\vspace{1em}
{\bf Acknowledgments:}
Most of this work was done during Robert Culling's time as James Borger's PhD student at the Australian National University, from 2015 through 2019. This work is presented in much more detail in Culling's thesis which can be found online \cite{}. Culling would like to acknowledge the support of the Mathematical Sciences Institute at ANU and the support from the Australian People through the Australian Government Research Training Program Stipend and Scholarship. Without this kindness the work would not have been completed.

\newpage
\section{Commutative Algebra over the Natural Numbers}

We present the following section on the commutative algebra of the natural numbers for sake of completeness. We follow the presentation of Bertrand To{\"e}n and Michel Vaqui{\'e} \cite{toenvaquie} and James Borger \cite{jbwittsemipos}. The reader may consult these authors for further details, along with the PhD thesis of the second author \cite{}. One may also like to consult Johnathon Golan \cite{golan} for a different perspective on monoids and semirings.

\begin{defn}
A commutative monoid is a triple $(M,+_{M},0_{0})$ which consists of: a set, $M$; a commutative and associative binary operation $+:M\times M \rightarrow M$ called addition, and the distinguished element $0 \in M$ such that $+(0,m) = m$ called the identity. If $(M,+_{M},0_{M})$ and $(N,+_{N},0_{N})$ are commutative monoids, then a homomorphism is a function of sets $\varphi:M \rightarrow N$ which maps $\varphi(0_{M}) = 0_{N}$ and commutes with the binary operations $+_{M}$ and $+_{N}$.
\end{defn}

We will denote a commutative monoid by its set and drop the subscripts from the binary operation and identity. Similarly we will write $a+b$ instead of $+(a,b)$. Throughout this paper commutative monoids will be referred to as $\bn$-modules and the category of $\bn$-modules will be denoted $\nmod$.

The category $\nmod$ has products and coproducts. If $I$ is a set indexing a family of $\bn$-modules $\{M_{i}\}_{I}$, then the set theoretic product $\prod_{I}M_{i}$ with component wise addition and the distinguished element $0 = (...,0,...)$ is the product in $\nmod$. It has the sub-$\bn$-module $\bigoplus_{I}M_{i} \subseteq \prod_{I}M_{i}$ of elements that have only finitely many components different from 0, this forms the coproduct of the family in $\nmod$.

If $M,N$ are $\bn$-modules, then the set $\homnmod(M,N)$ is an $\bn$-module under point-wise addition of homorphisms. Thus for a fixed $\bn$-module $M$, there is an endofunctor $\homnmod(M, -): \nmod \rightarrow \nmod$ on the category of $\bn$-modules. This functor has a left adjoint, $M\otimes -$. Thus for each $M,N$ in $\nmod$ there is an $\bn$-module $M\otimes N$ defined by the universal property of the left adjoint. This can be constructed in the same way as for $\bz$-modules, as long as one is careful to avoid negatives. See \cite{jbwittsemipos} for details.

\begin{defn}
A commutative $\bn$-algebra is defined to be a commutative monoid $(A,+_{A},0_{A})$ equipped with a commutative binary operation denoted $\times_{A}$ that distributes over $+_{A}$, with the property that for each $a \in A$ the equation $0_{A}\times_{A}a = 0_{A}$ is true, and has identity $1_{A}$. If $(A,+_{A},\times_{A},0_{A},1_{A})$ and $(B,+_{B},\times_{B},0_{B},1_{B})$ are commutative $\bn$-algebras, then a homomorphism of $\bn$-algebras is a homomorphism of monoids $\varphi:A \rightarrow B$ that commutes with the binary operations $\times_{A}$ and $\times_{B}$. If there exists a homomorphism $\varphi: A \rightarrow B$, then we say that $B$ is an $A$-algebra.
\end{defn}

Again, we will drop all subscripts and denote the $\bn$-algebras by their underlying set. All $\bn$-algebrs are assumed to be commutative for the rest of the paper, so we will drop the use of commutative. In the literature $\bn$-algebras are more commonly referred to as semirings. Notice that an $\bn$-algebra structure on a monoid is equivalent to a commutative monoid object in the monoidal category $(\nmod, \otimes)$. We will use $\nalg$ to denote the category of $\bn$-algebras. Note that $\bn$ is initial in $\nalg$, so there is no clash in the definitions of an $\bn$-algebra given above.

\begin{defn}
If $R$ is an $\bn$-algebra and $M$ a commutative monoid with a bilinear map $\cdot: R\times M \rightarrow M$ with the following associativity property: for each $r,s \in R$ and $m \in M$ the following equation holds $(rs)\cdot m = r\cdot(s\cdot m)$, then we say $M$ is an $R$-module. For two $R$-modules $M,N$ a commutative monoid morphism $f:M \rightarrow N$ is a morphism of $R$-modules if it satisfies each of the equations $f(rm) = rf(m)$, for each $r \in R$.
\end{defn}

We denote the category of $R$-modules by $\rmod$ and the $R$-module $R$ is both initial and final in this category. When $R = \bn$ the category of $\bn$-modules is equivalent to the category of commutative monoids, so there is no clash of notation.























\newpage
\section{Arithmetic Geometry over the Natural Numbers}

\begin{defn}
If $X: \nalg \rightarrow \catset$ is a representable functor, then we say $X$ is an affine $\bn$-scheme.
\end{defn}

We denote the category of $\bn$-schemes, with the natural transformations between them, by $\naff$. We note that $\naff$ is by definition a full and faithful subcategory of $\pre(\naff)$.

\begin{defn}
Let $X$ be a presheaf on $\naff$ and $(X_{i} \rightarrow X)_{i\in I}$ be a family of affine $\bn$-schemes over $X$. We say the family of $\bn$-schemes is a flat cover if (i) each morphism $X_{j}\rightarrow X$ is flat and (ii) the collection is faithful.
\end{defn}

\begin{defn}
Let $X$ be a presheaf on $\naff$ and $(X_{i}\rightarrow X)_{i \in I}$ a flat cover of $X$. If there exists some finite set $J \subseteq I$ such that $(X_{j}\rightarrow X)_{j\in J}$ is a flat cover of $X$, then we say $(X_{i}\rightarrow X)_{i \in I}$ is an fpqc cover of $X$.
\end{defn}

Bertrand To{\"e}n and Michel Vaqui{\'e} show the fpqc coverings form a Grothedieck topology and that the affine $\bn$-schemes are sheaves on the site ($\naff$,fpqc) \cite{toenvaquie}. In light of their work we see the following full and faithful embeddings of subcategories $\naff \subseteq \shf(\naff) \subseteq \pre(\naff)$. Therefore we should consider the geometry of the natural numbers to include these broader categories of sheaves and presheaves.

\begin{defn}
If $Y \rightarrow X$ is a morphism of presheaves on $\naff$, then we say $Y \rightarrow X$ is totally split if there exists a finite set $I$ such that $Y \cong \coprod_{I}X$.
\end{defn}

We say that a morphism of split schemes is split if it is defined by a set map.

\begin{defn}

If $Y \rightarrow X$ and $Z \rightarrow X$ are totally split morphisms, then we say that morphism $f: Y \rightarrow Z$ is

\end{defn}

We say a morphism is finite \'{e}tale if it is fpqc locally totally split.

\begin{defn}
Let $Y \rightarrow X$ be a morphism of presheaves on $\naff$. If there exists an fpqc cover $(U_{i}\rightarrow X)_{i\in I}$ and a family of finite sets $N_{i}$ such that for each $i$ there exists an isomorphism $Y \times_{X} U_{i} \cong \coprod_{I} U_{i}$, then we say the morphism $Y \rightarrow X$ is finite \'{e}tale.
\end{defn}



If $Y,Z \rightarrow X$ are finite \'{e}tale morphisms, then we may assume there is a cover of $(U_{i}\rightarrow X)_{i \in I}$ that splits both $Y,Z$ and that any morphism $f: Y \rightarrow Z$ is

\begin{lemma}
If $\varphi:Y \rightarrow X$ and $\psi: Z \rightarrow X$ are finite \'{e}tale morphisms with a morphism $f: Y \rightarrow X$, then there exists an fpqc cover $(U_{i} \rightarrow X)$ such that (i) $Y\times_{X}U_{i} \cong \coprod_{N_{i}}U_{i}$ (ii) $Z\times_{X}U_{i} \cong \coprod_{M_{i}}U_{i}$ for some finite sets $N_{i}, M_{i}$.

Furthermore the morphism $f_{i}: \coprod_{N_{i}}U_{i} \rightarrow \coprod_{M_{i}}U_{i}$ induced by the pullback is determined by a set map $\alpha_{i}: N_{i} \rightarrow M_{i}$.
\end{lemma}












\newpage
\section{Galois Category of Finite \'{E}tale Morphisms}

\newpage
\section{Non-Negative Reals and Real Number Fields}

\newpage
\section{Natural Numbers}

\newpage
\section{Non-Trivial Quadratic Finite \'{E}tale Cover}

\newpage
\printbibliography
\vspace{2em}
\end{document}
